\chapter*{Introduction}
\addcontentsline{toc}{chapter}{Introduction}

\paragraph{}Dans le cadre de notre projet long de la deuxième année du cursus ingénieur à Supélec, nous devons développer un logiciel de simulation de circuit électrique. Il s'agit d'un logiciel qui permet à l'utilisateur de créer graphiquement un circuit électrique (à l'aide de composants génériques ou personnalisés) puis de simuler son fonctionnement dans différentes conditions. Un logiciel commercial de ce type parmi les plus connus est LTspice.

\paragraph{}Ce projet ayant déjà été proposé l'an dernier, il a été décidé que nous commencerions à partir de zéro afin de bien maîtriser chaque étape de la conception du logiciel. Cependant, nous conservons le langage de programmation précédemment utilisé. Le logiciel sera donc codé en C\#. Il s'agit d'un langage orienté objet, très inspiré du C++, mais développé par Microsoft dans le cadre de la plateforme .NET et présentant de nombreuses différences. Pour gérer le front-end, nous utiliserons WPF (\textit{Windows Presentation Foundation}).

\paragraph{}D'un point de vue organisationnel, le projet se déroulera en plusieurs phases : tout d'abord une phase de montée en compétences en codant un petit jeu, puis l'implémentation d'une interface graphique basique pour le programme. Cela nous permettra ensuite d'ajouter la simulation du circuit. S'il nous reste du temps, nous pourrons ajouter des fonctionnalités au programme pour le rendre plus agréable, intuitif et rapide à utiliser.

