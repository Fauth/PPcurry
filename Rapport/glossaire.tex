\newglossaryentry{IDE}
{
	name=IDE,
	description={L'environnement de développement intégré (IDE en anglais) est un logiciel permettant à la fois d'éditer du code et de le compiler. Il comprend généralement des outils simplifiant le développement, tels une coloration syntaxique, une auto-complétion du code ou un débogueur.}
}

\newglossaryentry{frontend}
{
	name=Front-end,
	description={Le front-end représente la partie d'un logiciel (et donc le code associé) qui interagit avec l'utilisateur. Il s'agit donc de l'interface entre le back-end et l'utilisateur.}
}

\newglossaryentry{backend}
{
	name=Back-end,
	description={Le back-end représente la partie d'un logiciel (et donc le code associé) qui gère la logique et le fonctionnement interne d'un logiciel.}
}

